\documentclass[Bachelorarbeit.tex]{subfiles}
\begin{document}
\chapter{Einführung}

Diese Arbeit soll die Erkenntnisse dokumentieren, welche ich im Zuge des Berufspraktikums  bei der Firma \textbf{smog.at} GmbH im Sommer 2013 erwerben durfte.
Die Firma \textbf{smog.at} GmbH ist ein junges, innovatives und aufstrebendes Unternehmen mit Firmensitz in Lauterach. Zu ihren Tätigkeitsfeldern zählen unter anderem: Netwerk- und Kommunikations-Technologien, Sicherheits- und Risiko-Management sowie Zertifizierung von IT-Umgebung.\\
\begin{comment}
Der Hauptaufgabenbereich meines Praktikums bestand darin, verschiedene Kommunikationskanäle wie beispielsweise \acf{UMTS}, \acf{RFID} und Modbus auf verschiedenen \acf{SOC}-Modellen zu implementieren und das Vorgehen reproduzierbar zu dokumentieren.
\end{comment}
Der Hauptaufgabenbereich meines Praktikums bestand darin, die Kommunikationsprotokolle \acf{UMTS}, \acf{RFID} und Modbus auf verschiedenen \acf{SOC}-Modellen zu implementieren und das Vorgehen reproduzierbar zu dokumentieren.
 Die folgenden Abschnitte stellen eine Übersicht der verschiedenen, von mir bearbeiteten Projekte da.

\section*{Datenverbindung via \ac{UMTS}}
Der Inhalt dieses Projektes ist es, eine Datenverbindung für ein bestehendes Server-Monitoring-Systems zu realisieren. 
Der Nutzen des Systems besteht darin, den Zustand der registrierten Dienste, wie zum Beispiel der eines Webservers, zu überwachen. 
Sollte bei einer registrierten Komponente ein kritischer Zustand eintreten, besteht die Möglichkeit, auf verschiedene Wegen die definierten Verantwortlichen zu benachrichtigen. \\
Hierbei liegt die Aufgabenstellung nicht darin, die Überwachungssoftware selbst zu entwickeln, sondern vielmehr die Einbettung und Anbindung eines \ac{UMTS}-Modems an das Betriebssystem und dessen Dienste sicherzustellen. 
Als Grundlage für diese Aufgabe dient ein \ac{SOC}.
\begin{comment}
Software-seitig wird CentOS als Betriebssystem verwendet.
\end{comment} 
Als Betriebssystem wird CentOS verwendet.\\
\\
Folgende Punkte sind im Rahmen der Spezifikation vorgegeben:\\
Um die Verfügbarkeit zu erhöhen, soll neben den Konfigurationen, wie zum Beispiel Zugang zum Netzwerk, eine redundante Netzwerkschnittstelle mit Hilfe eines \ac{UMTS}-Modem realisiert werden. 
Diese soll des weiteren auch für den automatisierten Versand von SMS, sowie im Falle eines Netzwerkausfalls als Zugriffsmöglichkeit für Wartungsarbeiten verwendet werden. 
Zusätzlich darf die \ac{UMTS}-Verbindung während des automatisierten Versands von SMS nicht unterbrochen werden. \\
Für den sicheren Zugang zu geschlossenen Netzwerken soll OpenVPN eingesetzt werden. 
Des Weiteren sollen Wartungszugriffe auf das Gerät über das \ac{SSH}-Protokoll  erfolgen.

\section*{Kommunikation via \\ \ac{RFID}}
In diesem Projekt soll - mit minimalen Hardware-Ressourcen - eine Grundlage für die Entwicklung und Einbettung eines RFID-Shield in einem Linux-Betriebssystem realisiert sowie für zukünftige Einsatzmöglichkeiten dokumentiert werden. 
\begin{comment}
Ein weiteres Ziel besteht darin, dass die Umsetzung bis auf die Limitierung, dass ein Linux-Betriebssystem eingesetzt werden soll, Plattform unabhängig ist. 
\end{comment}
Ein weiteres Ziel besteht darin, dass die Umsetzung auf beliebigen Linux-Betriebssystemen eingesetzt werden kann.
Dies soll die Grundlage dafür bilden, weitere Applikationen zu entwickeln, welche auf den dokumentierten Konfigurationen aufbauen. \\
Neben einem \ac{SOC} wird ein "`Philips PN532/C1"' als \ac{RFID}-Komponente eingesetzt.
Als Betriebssystem wird eine für \acs{ARM}-Architekturen portierte Debian-Distribution verwendet. 
Die Kommunikation wird über die frei erhältliche \texttt{libnfc}-Bibliothek realisiert.

\section*{Visual Energy}
Bei dem Projekt Visual Energy handelt es sich um eine Konzeptlösung, bestehend aus Hard- und Softwarekomponenten.\\
In erster Linie soll der Funktionsumfang das Auslesen und Speichern von gemessenen Verbrauchswerten umfassen.
Dabei werden die Messwerte eines Energiezählers, wie zum Beispiel die aktuelle Spannung, ausgelesen und weiter verarbeitet.\\
Das Ziel ist es, die ermittelten Daten dem/der NutzerIn auf eine einfache Art zur Verfügung zu stellen.
Die Verwendung von Visual Energy gibt den AnwenderInnen einen transparenten Einblick in den persönlichen Energieverbrauch. 
Durch eine Präsentation der Verbrauchsanalyse soll eine Sensibilisierung stattfinden und im besten Fall gar eine Bewusstseinsbildung zum nachhaltigen und verantwortungsvollen Umgang mit der Ressource Energie anstoßen.\\
\\\\
Da es den Rahmen sprengen würde, alle drei Projekte ausführlich zu beschreiben nimmt diese Arbeit in erster Linie Bezug auf das Projekt Visual Energy.
%
%\section*{Struktur der Arbeit}
%
%Vor dem Beginn der State of the Art-Recherche wurden die Rahmenbedingung definiert und soweit möglich, ausschließlich Hard- und Softwarekomponenten betrachtet, die zu den Rahmenbedingungen passen.
%Darauf aufbauend, wurde eine Analyse durchgeführt in der, unter anderem, die Ergebnisse und Begründungen für die ausgewählten Technologien - im Hard- und Software-Bereich - definiert wurden. 
%Auf Basis der bestehenden Technologien werden Überlegungen zur Softwarearchitektur erläutert und anschließend in ein Konzept-Modell umgesetzt.
%Bei der "`Implementierung"' wird als Schwerpunkt die Kommunikation zwischen den Energiezählern und dem \ac{SOC} betrachtet. 
%Abgeschlossen wird die Arbeit durch die Evaluation. 
%An dieser Stelle werden die Ergebnisse betrachtet, ein Ausblick auf weitere Entwicklungsmöglichkeiten gegebenen sowie ein Fazit des Projektes diskutiert.
%Im Anhang finden neben UML-Diagrammen auch der Befehlssatz für das entwickelte Programm ComModbus. 
 

\end{document}